\documentclass[12pt,a4paper]{letter}
\usepackage[utf8]{inputenc}
\usepackage{amsmath}
\usepackage{amsfonts}
\usepackage{amssymb}
\author{Fabian Fest}
% Headers
\usepackage{fancyhdr}
\pagestyle{fancy}
\thispagestyle{empty}
\fancyhead[LO]{Donald E. Knuth - his typesetting system - \TeX}
\fancyhead[RE]{Fabian Fest, FIA 53}

\title{Knuth and his typesetting system - \TeX} 


\begin{document} 
	\begin{letter}{} 
		\opening{Donald E. Knuth - his typesetting system - \TeX} 
			
		In the past, around 40 years ago, there was no Microsoft Office package to write formated letter or books, especially for scientific texts or mathematical formulas there was no good and easy way to write them down.
		\\
		Donald E. Knuth, an American computer scientist, mathematician and professor emerithus at Stanford University, writer of some scientific book around programming style, e.g. "The Art of Computer Programming" where he describes the interconnection between readability of code and its documentation written in the same document.
		\\
		Knuth was born in Milwaukee, Wisconsin, as son of a teacher, who has a little printing shop. He choose physics over music and began studying physics. He was early introduced in mainframes and learned to write assembler. Later he switched from physics to mathematics and received his bachelor of mathematics in 1960. Simultaneously being given a master of science degree by special award of the faculty who considered his work exceptionally outstanding. 
		\\
		In the following years Knuth wrote books about programming, especially the art of programming.
		Just before publishing the first volume of The Art of Computer Programming, Knuth left Caltech to accept employment with the Institute for Defense Analyses' Communications Research Division, then situated on the Princeton University campus, which was performing mathematical research in cryptography to support the National Security Agency.
		\\
		From 1971 Knuth was awarded many times for his great work in mathematics and informatics. He also received many other prices for his scientific work.
		\\\\
		\TeX \space become a very impressive tool to create nice formated articles, letter, documentation even books. Today there are many forks, based on Knuth's \space \TeX \space like \LaTeX, \LaTeXe \space or many more.
		
		
		\closing{By the way, this essay is written with \LaTeX.} 
		%\cc{Cclist} 
		%\ps{adding a postscript} 
		%\encl{list of enclosed material} 
	\end{letter} 
\end{document}